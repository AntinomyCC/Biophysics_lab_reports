\documentclass[12pt]{article}
\begin{document}
\section{material}
Protein we actually using: BSA

Volume we finally got is around 10 microliter (which should be 20 microliter)
reason: some of them left in the column.

\section{How likely to find proteins with different labels}
The probability of a protein has k dye labels follows follows Poisson distribution:\\

\(P(k,\lambda) = \frac{\lambda^k e^{-\lambda}}{k!}\)\\

where $\lambda$ is the average of the labelled protein.

\textbf{How to acquire $\lambda$}

Refer to the \textit{Characterization of labelling(P9)}, the $\lambda$ here is the labelling efficiency(the ratio of dye molecules to protein molecule), which can be acquired by the process on P8 and P9.

After that, the probability of non-labelled, single labelled and double labelled are derived a $P(k = 0), P(k = 1) and P( k = 2)$

\end{document}